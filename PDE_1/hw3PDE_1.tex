%  This is a LaTex file.

%  Homework for the course "AMath 585:  Applied Linear Algebra and Numerical Analysis",
%  Autumn quarter, 2009, Anne Greenbaum.


%   A latex format for making homework assignments.


\documentclass[letterpaper,12pt]{article}

%          The page format, somewhat wider and taller page than in art12.sty.

\topmargin -0.1in \headsep 0in \textheight 8.9in \footskip 0.6in
\oddsidemargin 0in  \evensidemargin 0in  \textwidth 6.5in
\usepackage{graphicx}
\usepackage{listings}
\usepackage{caption}
\usepackage{subcaption}
\usepackage{color}
\usepackage{float}
\definecolor{keywords}{RGB}{255,0,90}
\definecolor{comments}{RGB}{0,0,113}
\definecolor{red}{RGB}{160,0,0}
\definecolor{green}{RGB}{0,150,0}
\definecolor{codegreen}{rgb}{0,0.6,0}
\definecolor{codegray}{rgb}{0.5,0.5,0.5}
\definecolor{codepurple}{rgb}{0.58,0,0.82}
\definecolor{backcolour}{rgb}{0.95,0.95,0.92}
\definecolor{brown}{rgb}{0.59, 0.29, 0.0}
\definecolor{beaublue}{rgb}{0.74, 0.83, 0.9}
\definecolor{orange}{rgb}{1.0, 0.5, 0.0}
\definecolor{darkslategray}{rgb}{0.18, 0.31, 0.31}
\definecolor{deepblue}{rgb}{0,0,0.5}
\definecolor{deepred}{rgb}{0.6,0,0}
\definecolor{deepgreen}{rgb}{0,0.5,0}
\lstdefinestyle{myMatlabstyle}{
	language=Matlab,
	backgroundcolor=\color{white},
	commentstyle=\color{codegreen},
	keywordstyle=\color{blue},
	%identifierstyle=\color{brown},
	numberstyle=\tiny\color{codegray},
	stringstyle=\color{orange},
	basicstyle=\footnotesize,
	breakatwhitespace=false,
	breaklines=true,
	captionpos=b,
	keepspaces=true,
	numbers=left,
	numbersep=5pt,
	showspaces=false,
	showstringspaces=false,
	showtabs=false,
	tabsize=2
}
\lstdefinestyle{myPythonstyle}{
	language=Python,
	basicstyle=\ttfamily\small,
	keywordstyle=\color{blue},
	backgroundcolor=\color{white},
	commentstyle=\color{green},
	stringstyle=\color{red},
	showstringspaces=false,
	%identifierstyle=\color{brown},
	breaklines=true,
}
\lstset{language=Matlab,frame=single}
\lstset{language=Python,frame=single}
\usepackage{amsmath}
\usepackage{epsfig}         % to insert PostScript figures
       % to insert PostScript figures

\begin{document}


%          Definitions of commonly used symbols.



%          The title and header.

\noindent
{\scriptsize ES$\_$APPM 411-1, Fall 2018} \hfill

\begin{center}
\large
Assignment 3.
\normalsize

Jithin D. George
\end{center}

\noindent
Due Oct 16
\vspace{.3in}

%           The questions!



\noindent


\begin{enumerate}
\item
\begin{enumerate}
\item
\[L u = [p(x)u'(x)]' +q(x)u(x)\]
\begin{align*} \int_a^b v L u  dx & =   \int_a^b v(x) ([p(x)u'(x)]' +q(x)u(x)) dx \\
& = [v(x) p(x)u'(x) ]_a^b+ \int_a^b  - v' p u' + q u v dx \\
& = [v pu'-v'pu ]_a^b+  \int_a^b v'' p u+ v'p'u + q u v dx \\
& = [v pu'-v'pu ]_a^b+ u \int_a^b v'' p + v'p' + q  v dx \\
\end{align*}
The adjoint equation is
\[L^* v = [p(x)v'(x)]' +q(x)v(x)\]
For the boundary conditions ,
\[v(b) p(b)u'(b)-v'(b)p(b)u(b)- v(a) p(a)u'(a)+v'(a)p(a)u(a)=0\]
\[u(b)(-\frac{\alpha_1}{\alpha_2} v(b) p(b)-v'(b)p(b))- u(a)(-\frac{\beta_1}{\beta_2}v(a) p(a)-v'(a)p(a))=0\]

Thus, the adjoint boundary conditions are
\[\alpha_1 v(b) p(b)+\alpha_2 v'(b)p(b)= 0\]
\[\beta_1 v(a) p(a)+ \beta_2 v'(a)p(a) = 0\]
\item
\[L u = [p(x)u'(x)]' +q(x)u(x)\]
\begin{align*} \int_a^b v L u  dx & =   \int_a^b v(x) ([p(x)u'(x)]' +q(x)u(x)) dx \\
& = [v(x) p(x)u'(x) ]_a^b+ \int_a^b  - v' p u' + q u v dx \\
& = [v pu'-v'pu ]_a^b+  \int_a^b v'' p u+ v'p'u + q u v dx \\
& = [v pu'-v'pu ]_a^b+ u \int_a^b v'' p + v'p' + q  v dx \\
\end{align*}
The adjoint equation is
\[L^* v = [p(x)v'(x)]' +q(x)v(x)\]
For the boundary conditions ,
\[v(b) p(b)u'(b)-v'(b)p(b)u(b)- v(a) p(a)u'(a)+v'(a)p(a)u(a)=0\]
\[p(b)u'(b)(v(b)-v(a))+u(a)(v'(a)p(a)-v'(b)p(b))=0\]
Thus, the adjoint boundary conditions are
\[v(b)-v(a) =0 \]
\[v'(a)p(a)-v'(b)p(b) =0 \]

\item
\[Lu = u''(x)+[c(x)-x]u'(x)+c'(x)u(x)\]
\begin{align*} \int_a^b v L u  dx & =   \int_a^b v(x) (u''(x)+[c(x)-x]u'(x)+c'(x)u(x) ) dx \\
& = [vu' + v(c-x)u ]_a^b+ \int_a^b  - v' u' -u(v(c'-1)+v'(c-x)) + c' u v dx \\
& = [vu' + v(c-x)u -v'u ]_a^b+ \int_a^b  v'' u -u(v(c'-1)+v'(c-x)) + c' u v dx \\
& = [vu' + v(c-x)u -v'u ]_a^b+ u \int_a^b  v''  -v(c'-1)+v'(c-x) + c'  v dx \\
& = [vu' + v(c-x)u -v'u ]_a^b+ u \int_a^b  v''  +v +v'(c-x)  dx \\
\end{align*}
The adjoint equation is
\[L^* v = v''(x) +(c(x)-x)v'(x)+v(x)\]
For the boundary conditions ,
\[v(1) u'(1)+v(1)(c(1)-1)u(1)-v'(1)u(1) - v(0) u'(0)- v(0)c(0)u(0)+v'(0)u(0)=0\]

\[-v(1)u(0) -v'(1)u(1) + v(0) u(1)+v'(0)u(0)=0\]
Thus, the adjoint boundary conditions are
\[v'(0)=v(1)\]
\[v'(1)=v(0)\]
\end{enumerate}

\item
\[x(1-x)y''-2y'+2y=0\]

\[y =x^r\sum_{n=0}^\infty a_nx^n = \sum_{n=0}^\infty a_nx^{r+n}= a_0 x^r + a_1 x^{r+1} +\ldots\]
with
\[a_0\neq 0\]

To find r, we plug $a_0 x^r $ into the ode.
\[r(r-1)x^{r-1}-r(r-1)x^r-2rx^{r-1}+2x^r=0\]
Equating the lowest order terms,
\[r^2-3r=0\]
\[ r = 3,0\]
So, the leading terms are 1 and $x^3$.
\[y = \sum_{n=0}^\infty a_n x^{n+r} \]
\[y' = \sum_{n=0}^\infty (n+r) a_{n} x^{n+r-1}\]
\[y''= \sum_{n=0}^\infty (n+r)(n+r-1) a_{n} x^{n+r-2} \]

\begin{align*}
\sum_{n=0}^\infty (n+r)(n+r-1) a_{n} x^{n+r-1} - \sum_{n=0}^\infty (n+r)(n+r-1) a_{n} x^{n+r} - 2 \sum_{n=0}^\infty (n+r) a_{n} x^{n+r-1} \\ + 2 \sum_{n=0}^\infty a_{n} x^{n+r}  =0
\end{align*}

To find the reccurence relation,
\begin{align*}
\sum_{n=0}^\infty (n+r+1)(n+r) a_{n+1} x^{n+r} - \sum_{n=0}^\infty (n+r)(n+r-1) a_{n} x^{n+r} - 2 \sum_{n=0}^\infty (n+r+1) a_{n+1} x^{n+r} \\ + 2 \sum_{n=0}^\infty a_{n} x^{n+r}  =0
\end{align*}
\[ a_{n+1} = a_n \frac{(n+r-1)(n+r) -2}{(n+r+1)(n+r) - 2 (n+r+1)}\]

When r =0,

\[ a_{n+1} = a_n \frac{n^2-n -2}{(n+1)(n-2)}=  a_n\]
When r =3,

\[ a_{n+1} = a_n \frac{(n+2)(n+3) -2}{(n+4)(n+3) - 2 (n+4)} = a_n \frac{n^2+5n+4}{n^2+5n+4}= a_n \]

So,

\begin{align*}
y &= c_1 (1+ x+x^2+x^3 +\hdots) + c_2 x^3 (1+ x+x^2+x^3 +\hdots)\\
&= c_1 \frac{1}{1-x} + c_2 \frac{x^3}{1-x}
\end{align*}

\item
\[y'' - xy = 0\]
This is not singular. So,
\[y = \sum_{n=0}^\infty a_n x^n \]

\[y''= \sum_{n=0}^\infty (n+2)(n+1) a_{n+2} x^{n} \]
\[\sum_{n=0}^\infty (n+2)(n+1) a_{n+2} x^{n} + \sum_{n=0}^\infty a_n x^{n+1} =0 \]
\[2a_2 + \sum_{n=1}^\infty (n+2)(n+1) a_{n+2} x^{n} + \sum_{n=1}^\infty a_{n-1} x^{n} =0 \]
So,
\[a_2 = 0\]
\[a_{n+2} = -\frac{1}{(n+2)(n+1)}a_{n-1}\]
\[y = a_0\big(1 -  \frac{x^3}{6} +  \frac{x^6}{30} + \ldots \big) +a_1\big(x - \frac{x^4}{12}+ \frac{x^7}{3*4*6*7} + \ldots \big)\]
\end{enumerate}
\end{document}
