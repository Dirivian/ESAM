%  This is a LaTex file.

%  Homework for the course "AMath 585:  Applied Linear Algebra and Numerical Analysis",
%  Autumn quarter, 2009, Anne Greenbaum.


%   A latex format for making homework assignments.


\documentclass[letterpaper,12pt]{article}

%          The page format, somewhat wider and taller page than in art12.sty.

\topmargin -0.1in \headsep 0in \textheight 8.9in \footskip 0.6in
\oddsidemargin 0in  \evensidemargin 0in  \textwidth 6.5in
\usepackage{graphicx}
\usepackage{listings}
\usepackage{caption}
\usepackage{subcaption}
\usepackage{color}
\usepackage{float}
\definecolor{keywords}{RGB}{255,0,90}
\definecolor{comments}{RGB}{0,0,113}
\definecolor{red}{RGB}{160,0,0}
\definecolor{green}{RGB}{0,150,0}
\definecolor{codegreen}{rgb}{0,0.6,0}
\definecolor{codegray}{rgb}{0.5,0.5,0.5}
\definecolor{codepurple}{rgb}{0.58,0,0.82}
\definecolor{backcolour}{rgb}{0.95,0.95,0.92}
\definecolor{brown}{rgb}{0.59, 0.29, 0.0}
\definecolor{beaublue}{rgb}{0.74, 0.83, 0.9}
\definecolor{orange}{rgb}{1.0, 0.5, 0.0}
\definecolor{darkslategray}{rgb}{0.18, 0.31, 0.31}
\definecolor{deepblue}{rgb}{0,0,0.5}
\definecolor{deepred}{rgb}{0.6,0,0}
\definecolor{deepgreen}{rgb}{0,0.5,0}
\lstdefinestyle{myMatlabstyle}{
	language=Matlab,
	backgroundcolor=\color{white},
	commentstyle=\color{codegreen},
	keywordstyle=\color{blue},
	%identifierstyle=\color{brown},
	numberstyle=\tiny\color{codegray},
	stringstyle=\color{orange},
	basicstyle=\footnotesize,
	breakatwhitespace=false,
	breaklines=true,
	captionpos=b,
	keepspaces=true,
	numbers=left,
	numbersep=5pt,
	showspaces=false,
	showstringspaces=false,
	showtabs=false,
	tabsize=2
}
\lstdefinestyle{myPythonstyle}{
	language=Python,
	basicstyle=\ttfamily\small,
	keywordstyle=\color{blue},
	backgroundcolor=\color{white},
	commentstyle=\color{green},
	stringstyle=\color{red},
	showstringspaces=false,
	%identifierstyle=\color{brown},
	breaklines=true,
}
\lstset{language=Matlab,frame=single}
\lstset{language=Python,frame=single}
\usepackage{amsmath}
\usepackage{epsfig}         % to insert PostScript figures
       % to insert PostScript figures

\begin{document}


%          Definitions of commonly used symbols.



%          The title and header.

\noindent
{\scriptsize ES$\_$APPM 411-1, Fall 2018} \hfill

\begin{center}
\large
Assignment 1.
\normalsize

Jithin D. George
\end{center}

\noindent
Due Oct 8
\vspace{.3in}

%           The questions!



\noindent


\begin{enumerate}
\item
\[x^2y'' + 7xy' +13y =0\]
This is an equidimensional equation. Plugging in $x^m$
\[m(m-1)+7m+13=0\]
\[m^2+6m+13=0\]
\[m = -3 +2i,-3 -2i\]
So, the solution is
\[y = c_1 x^{-3}cos(2x)+ c_2 x^{-3}sin(2x)\]

\item
\[y''-3y'+2y=xe^x+xe^{-x}\]
The homogeneous solution is $c_1 e^x +c_2 e^{2x}$
The particular solution should  have the format
\[ae^x+be^{-x}+ c xe^x+d xe^{-x}\]
But $e^x$ is part of the homogeneous solution. So, we replace it by $x^2e^x$ since $xe^x$ is already part of our guess. Our new ansatz is
\[ax^2e^x+be^{-x}+ c xe^x+d xe^{-x}\]
Plugging it into the solution, we have
\begin{align*}
ax^2e^x+ 4axe^x+2ae^x+be^{-x}+2ce^x-2de^{-x}+ c xe^x+d xe^{-x}\\
-3cxe^x-3ce^x-3ax^2e^x-6axe^x+3be^{-x}+3dxe^{-x}-3de^{-x}\\
+2ax^2e^x+2be^{-x}+2c xe^x+2d xe^{-x}
\\ = xe^x+xe^{-x}
\end{align*}
\[2a+2c-3c=0\to c=2a\]
\[b-2d-3d+3b+2b=0 \to 6b =5d\]
\[4a+c-3c-6a+2c=1\to a = -\frac{1}{2}\]
\[d+3d+2d=1 \to d = \frac{1}{6}\]

Thus, the general solution is
\[y = c_1 e^x +c_2 e^{2x} -\frac{1}{2}x^2e^x+\frac{5}{6}e^{-x}- xe^x+\frac{1}{6} xe^{-x}\]

\item

\[y'' +y = (e^x+1)\sin x\]
The homogeneous solution is
\[y = c_1 \sin x + c_2 \cos x\]
Since sin and cos appear in the homogeneous solution, our ansatz for the particular solution is
\[y_p = a x \sin x + b x \cos x + c e^x \sin x + d e^x \cos x \]
\[y_p' = a  \sin x + a x \cos x + b \cos x - b x \sin x + c e^x \sin x+ c e^x \cos x + d e^x \cos x-  d e^x \sin x \]
\[y_p'' = 2a  \cos x - a x \sin x -2 b \sin x - b x \cos x +  2c e^x \cos x - 2 d e^x \sin x \]
Plugging it into the ode, we get
\[a =0, b = -\frac{1}{2}, c= \frac{1}{5}, d=-\frac{2}{5}\]
So, the general solution is
\[y = c_1 \sin x + c_2 \cos x -\frac{1}{2} x \cos x + \frac{1}{5} e^x \sin x -\frac{2}{5} e^x \cos x \]

\item

\[y''' -2y''+y'=1 +xe^x\]
This is an initial value problem with smooth coefficients and the first coefficient is 1. So, the solution exists everywhere and is unique.
Solving the homogeneous equation with the guess $e^{mx}$,
\[m^3-2m^2+m=0\]
\[m=0,1,1\]
Thus, the homogeneous solution is
\[y = c_1 +c_2 e^x + c_3 x e^x\]
The particular solution will have the form
\[y_p = a x + b x^2 e^x+c x^3e^x\]
\[y_p' = a  + (3c+b) x^2 e^x + 2bxe^x+c x^3e^x\]
\[y_p'' = (6c+b) x^2 e^x + (6c+4b)xe^x+2be^x+c x^3e^x\]
\[y_p''' =  (9c+b) x^2 e^x + (18c+6b)xe^x+(6c+6b)be^x+c x^3e^x\]
Plugging it into the ode, we have
\[a=1, c = \frac{1}{6}, b = -\frac{1}{2}\]
The general solution is
\[y = c_1 +c_2 e^x + c_3 x e^x+  x  -\frac{1}{2} x^2 e^x +\frac{1}{6} x^3e^x \]
Using the initial conditions,
\[c_1 + c_2 =0\]
\[c_2+c_3+1=0\]
\[c_2+2c_3-1=1\]

\[y = 4 -4 e^x + 3 x e^x+  x  -\frac{1}{2} x^2 e^x +\frac{1}{6} x^3e^x \]
\item
\[xy''- (x+1)y'+y=x^2e^{2x}, y(1)=0, y'(1)=e\]
This is an initial value problem starting at 1. The solution exists near 1 but there is a singularity at x reaches 0.

\[y_1 = x+1\]
\[y_2 = (x+1)u = x u +u\]
\[y_2' =  x u' +u'+u\]
\[y_2'' = u' +xu'' +u''+u'= 2u'+(x+1)u''\]
Plugging this into the ode, we have
\[2xu'+x(x+1)u''-(x+1)^2u'-(x+1)u'+(x+1)u=x^2e^{2x}\]
\[2xu'+x(x+1)u''-(x+1)^2u'=x^2e^{2x}\]
\[u''-\bigg(\frac{x+1}{x}-\frac{2}{x+1}\bigg)u'=\frac{xe^{2x}}{x+1}\]
\[e^{-\bigg(\frac{x+1}{x}-\frac{2}{x+1}\bigg)}u''-e^{-\bigg(\frac{x+1}{x}-\frac{2}{x+1}\bigg)}\bigg(\frac{x+1}{x}-\frac{2}{x+1}\bigg)u'=e^{-\bigg(\frac{x+1}{x}-\frac{2}{x+1}\bigg)}\frac{xe^{2x}}{x+1}\]
\[\bigg(e^{\int-\bigg(\frac{x+1}{x}-\frac{2}{x+1}\bigg)dx}u'\bigg)'=e^{\int-\bigg(\frac{x+1}{x}-\frac{2}{x+1}\bigg)dx}\frac{xe^{2x}}{x+1}\]
\[e^{-x}\frac{(x+1)^2}{x}u'= \int e^{-x}\frac{(x+1)^2}{x}\frac{xe^{2x}}{x+1} dx= \int (x+1)e^{x} dx  \]
\[e^{-x}\frac{(x+1)^2}{x}u'= xe^x+ c_1 \]
\[u'= \frac{x^2e^{2x}}{(x+1)^2}+ c_1 \frac{xe^{x}}{(x+1)^2} \]
\[u'= \frac{e^{2x}(x-1)}{2(x+1)}+c_1 \frac{e^x}{x+1}+c_2\]

\[y(x)= \frac{1}{2}e^{2x}(x-1) +c_1 e^x + c_2 (x+1)\]
 Plugging in initial conditions,
 \[c_1 e +2c_2=0\]
 \[\frac{e^2}{2}+c_2+c_1e=e\]
 \[c_1 = 2-e, c_2 =\frac{e^2}{2}-e\]
 \[y(x)= \frac{1}{2}e^{2x}(x-1) +(2-e) e^x + (\frac{e^2}{2}-e) (x+1)\]
 \item
 \[(x-1)y''-xy'+y=(x-1)^2\]
 \[y = x u\]
 \[y' = x u' + u\]
  \[y'' = x u'' + 2u'\]
	Putting this into the homogeneous ode,
	\[ x(x-1)u''+2(x-1)u'-x^2u'-xu+xu=0\]
		\[ x(x-1)u''=(x^2-2x+2)u'\]

			\[u'=   c_1\frac{e^x(1-x)}{x^2} \]
				\[u= -c_1 \frac{e^x}{x} + c_2 \]
				\[y = - c_1 e^x + c_2 x\]

	From the initial conditions, we get $c_1 =0, c_2 = 0 $
 This is a trivial solution for the homogeneous problem. So, the inhomogeneous problem has a unique solution.

 Plugging the derivatives into the inhomogeneous ode, we get
 	\[ x(x-1)u''-(x^2-2x+2)u' = (x-1)^2\]
	 	\[ u''-\frac{(x^2-2x+2)}{x(x-1)}u' = 1 - \frac{1}{x}\]
			\[ (\frac{e^{-x}x^2u'}{1-x})' =-e^{-x}x\]
		\[ \frac{e^{-x}x^2u'}{1-x} =e^{-x}x+e^{-x}+c_1 \]
				\[ u'=\frac{1}{x^2} -1 +c_1 \frac{e^{-x}(1-x)}{x}\]
								\[ u =-\frac{1}{x} -x +c_1 \frac{e^{x}}{x}+c_2\]
								\[ y =-1 -x^2 +c_1 e^{x}+c_2\]
				Plugging in the initial conditions, we get
				\[ y =-1 -x^2 +\frac{1}{4\sqrt{e}-1} e^{x}+\frac{4\sqrt{e}-5}{4\sqrt{e}-1}\]
 \item
\[ 4 y'' +y=x , y(\pi)= y(-\pi)=0\]
The homogeneous solution is
\[y =  c_1 \cos(\frac{x}{2})\]
in addition to the trivial solution. This means that the inhomogeneous equation has either no solution or infinitely many solutions.

Checking the solvability condition,
\[\int_{-\pi}^{\pi } x c_1 \cos(\frac{x}{2}) = 0\]
Thus, the BVP is solvable and has infinitely many solutions. $x$ is a particular solution. So, the general solution is
\[y =  c_1 \cos(\frac{x}{2}) + c_2 \sin(\frac{x}{2})+ x \]
From the BCs, we get $c_2 = - \pi$.
\[y =  c_1 \cos(\frac{x}{2}) -\pi  \sin(\frac{x}{2})+ x \]


 \item
  This is an initial value problem where the coefficients of the derivatives are smooth. So, the solution exists and is unique.

	The general solution is
	\[y = -\cos x +c_1 x +c_2\]
	From the initial conditions, we can find that
	\[c_2 =1, c_1 =0\]
	Thus, the solution is
	\[y = -\cos x +1\]
	\item
	\[9y''+y = xe^{-x^2}, y(-3\pi)=y(3\pi)=0\]
	The homogeneous solution is
	\[y = c_1 \sin(\frac{x}{3})\]
	in addition to the trivial solution. This means that the inhomogeneous equation has either no solution or infinitely many solutions.
Checking the solvability condition,
	\[\int_{-3\pi}^{3\pi } x e^{-x^2}c_1 \sin(\frac{x}{3}) dx =18 c_1 \int_{0}^{\pi } u e^{-9u^2}\sin(u) du > 0  \]
	Hence, there is no solution for the inhomogeneous ode.
\end{enumerate}
\end{document}
