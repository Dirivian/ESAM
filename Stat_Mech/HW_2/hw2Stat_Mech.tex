%  This is a LaTex file.

%  Homework for the course "AMath 585:  Applied Linear Algebra and Numerical Analysis",
%  Autumn quarter, 2009, Anne Greenbaum.


%   A latex format for making homework assignments.


\documentclass[letterpaper,12pt]{article}

%          The page format, somewhat wider and taller page than in art12.sty.

\topmargin -0.1in \headsep 0in \textheight 8.9in \footskip 0.6in
\oddsidemargin 0in  \evensidemargin 0in  \textwidth 6.5in
\usepackage{graphicx}
\usepackage{listings}
\usepackage{caption}
\usepackage{subcaption}
\usepackage{color}
\usepackage{float}
\usepackage{tikz}
\usetikzlibrary{decorations.pathreplacing}
\definecolor{keywords}{RGB}{255,0,90}
\definecolor{comments}{RGB}{0,0,113}
\definecolor{red}{RGB}{160,0,0}
\definecolor{green}{RGB}{0,150,0}
\definecolor{codegreen}{rgb}{0,0.6,0}
\definecolor{codegray}{rgb}{0.5,0.5,0.5}
\definecolor{codepurple}{rgb}{0.58,0,0.82}
\definecolor{backcolour}{rgb}{0.95,0.95,0.92}
\definecolor{brown}{rgb}{0.59, 0.29, 0.0}
\definecolor{beaublue}{rgb}{0.74, 0.83, 0.9}
\definecolor{orange}{rgb}{1.0, 0.5, 0.0}
\definecolor{darkslategray}{rgb}{0.18, 0.31, 0.31}
\definecolor{deepblue}{rgb}{0,0,0.5}
\definecolor{deepred}{rgb}{0.6,0,0}
\definecolor{deepgreen}{rgb}{0,0.5,0}
\lstdefinestyle{myMatlabstyle}{
	language=Matlab,
	backgroundcolor=\color{white},
	commentstyle=\color{codegreen},
	keywordstyle=\color{blue},
	%identifierstyle=\color{brown},
	numberstyle=\tiny\color{codegray},
	stringstyle=\color{orange},
	basicstyle=\footnotesize,
	breakatwhitespace=false,
	breaklines=true,
	captionpos=b,
	keepspaces=true,
	numbers=left,
	numbersep=5pt,
	showspaces=false,
	showstringspaces=false,
	showtabs=false,
	tabsize=2
}
\lstdefinestyle{myPythonstyle}{
	language=Python,
	basicstyle=\ttfamily\small,
	keywordstyle=\color{blue},
	backgroundcolor=\color{white},
	commentstyle=\color{green},
	stringstyle=\color{red},
	showstringspaces=false,
	%identifierstyle=\color{brown},
	breaklines=true,
}
\lstset{language=Matlab,frame=single}
\lstset{language=Python,frame=single}
\usepackage{amsmath}
\usepackage{epsfig}         % to insert PostScript figures
       % to insert PostScript figures

\begin{document}


%          Definitions of commonly used symbols.



%          The title and header.

\noindent
{\scriptsize ME 422, Fall 2018} \hfill

\begin{center}
\large
Assignment 1.
\normalsize

Jithin D. George
\end{center}

\noindent
Due Oct 17
\vspace{.3in}

%           The questions!



\noindent


\begin{enumerate}
\item

There are $2^{100} \approx 10^{30}$ configurations possible. I have an i3 processor which has roughly 3 GHz. This means it can do $3x10^9$ calculations in a second. So, it would take around $10^{20}$ seconds to do calculations of all the configurations. That's $10^{13}$ years of calculations!.

\item

\begin{enumerate}
\item

\begin{tikzpicture}




\draw[->](3.5,-0.7) -- (3.5,0) ;
\draw[->](4,-0.7) -- (4,0) ;
\node at (4.25,-0.7) {,};
\draw[->](4.5,0) -- (4.5,-0.7) ;
\draw[->](5,-0.7) -- (5,0) ;
\node at (5.25,-0.7) {,};
\draw[->](5.5,-0.7) -- (5.5,0) ;
\draw[->](6,0) -- (6,-0.7) ;

\node at (6.25,-0.7) {,};
\draw[->](6.5,0) -- (6.5,-0.7) ;
\draw[->](7,0) -- (7,-0.7) ;

\end{tikzpicture}

\item

\[Z = \sum_i e^{-\beta \epsilon U_i} = 2 e^{\beta \epsilon}+ 2 e^{-\beta \epsilon} = 4 cosh(\beta \epsilon)\]
 \item
 There are 4 states. So, the mean energy is
 \[<E>= \frac{2\epsilon e^{-\beta \epsilon} -2\epsilon e^{ \beta \epsilon} }{Z}= -\frac{2\epsilon e^{\beta \epsilon} -2\epsilon e^{- \beta \epsilon} }{2 e^{\beta \epsilon}+ 2 e^{-\beta \epsilon}} = - \epsilon \tanh(\beta \epsilon) \]
 \item Mean energy is close to $-\epsilon$ when $\tanh(\beta \epsilon)$ close to 1, which happens for $\beta \epsilon$ sufficiently greater than 0. This happens because the parallel spin terms dominate then giving the mean energy equivalent to that of a parallel spin.
 \item One.
 \[\lim_{T \to 0} \frac{2 e^{\beta \epsilon}}{2 e^{\beta \epsilon}+2 e^{-\beta \epsilon}} = \lim_{\beta  \to \infty} \frac{2 e^{\beta \epsilon}}{2 e^{\beta \epsilon}+2 e^{-\beta \epsilon}} = 1 \]
 \item Zero.
  \[\lim_{T \to 0} \frac{2 e^{-\beta \epsilon}}{2 e^{\beta \epsilon}+2 e^{-\beta \epsilon}} = \lim_{\beta  \to \infty} \frac{2 e^{-\beta \epsilon}}{2 e^{\beta \epsilon}+2 e^{-\beta \epsilon}} = 0 \]
 \item Zero.
 \[\lim_{T \to \infty} <E> = \lim_{\beta  \to 0}  - \epsilon \tanh(\beta \epsilon) =0\]

 \item

Ratio of probability for the all up state to the all down state is 1 regardless of temperature because they have the same probability. Similarly, the ratio of the probablities of the two anti-parallel states are 1.



Ratio of the probability of a parallel state to an anti-parallel one is
 \[ \lim_{\beta  \to 0}\frac{ e^{\beta \epsilon}}{ e^{-\beta \epsilon}} = 1\]

 \item
 For the probabilities to be equal, we need
 \[e^{\beta \epsilon} = 2e^{-\beta \epsilon}\]
  \[2\beta \epsilon = \log(2)\]
	\[T = \frac{2 \epsilon}{k_B \log 2}\]

 \item

 It seems to be $\beta \epsilon$ that determines the probability of states and thus, $\frac{\epsilon}{k_B T}$ is the parameter that controls the nature of the system.

\end{enumerate}


\item

\begin{enumerate}
\item Each state is influenced only by its two neighbours. So, it's a closed system.

\item
 The minimum prossible energy is  $-4  \epsilon$.

 \item

 \begin{tikzpicture}




 \draw[->] (3.5,-0.7)-- (3.5,0) ;
 \draw[->](4,-0.7) -- (4,0) ;
 \node at (5,-0.7) { and };
 \draw[->](6,-1) -- (6,-1.7) ;
 \draw[->] (6.5,-1) --(6.5,-1.7)  ;


 \draw[->] (4,-1.7)--(4,-1)  ;
  \draw[->](3.5,-1.7) -- (3.5,-1) ;

	\draw[->](6.5,0) -- (6.5,-0.7) ;
  \draw[->](6,0)-- (6,-0.7)  ;
 \end{tikzpicture}

\end{enumerate}

\item
\begin{enumerate}

\item
idivide(a,b) returns the closest integer to a/b which is either lesser than or greater than or just the closest one to a/b depending on the option. Mod returns the remainder in a/b
\item It seems to be West, South, East, North.

\item From the Matlab output nbr, it seems to be periodic boundaries.
\item
\[[-1,-1,-1,-1] \to_{k=1} [1,-1,-1,-1] \to_{k=2} [1,1,-1,-1] \to_{k=1} [-1,1,-1,-1] \]

\end{enumerate}
\item

Tried running it for a 10x10 system. This is the result.

\begin{figure}[h!]
	\centering
	\includegraphics[width=9cm]{1hxny5.jpg}

\end{figure}

\end{enumerate}
\end{document}
