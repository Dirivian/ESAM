%  This is a LaTex file.

%  Homework for the course "AMath 585:  Applied Linear Algebra and Numerical Analysis",
%  Autumn quarter, 2009, Anne Greenbaum.


%   A latex format for making homework assignments.


\documentclass[letterpaper,12pt]{article}

%          The page format, somewhat wider and taller page than in art12.sty.

\topmargin -0.1in \headsep 0in \textheight 8.9in \footskip 0.6in
\oddsidemargin 0in  \evensidemargin 0in  \textwidth 6.5in
\usepackage{graphicx}
\usepackage{listings}
\usepackage{caption}
\usepackage{subcaption}
\usepackage{color}
\usepackage{float}
\usepackage{tikz}
\usetikzlibrary{decorations.pathreplacing}
\definecolor{keywords}{RGB}{255,0,90}
\definecolor{comments}{RGB}{0,0,113}
\definecolor{red}{RGB}{160,0,0}
\definecolor{green}{RGB}{0,150,0}
\definecolor{codegreen}{rgb}{0,0.6,0}
\definecolor{codegray}{rgb}{0.5,0.5,0.5}
\definecolor{codepurple}{rgb}{0.58,0,0.82}
\definecolor{backcolour}{rgb}{0.95,0.95,0.92}
\definecolor{brown}{rgb}{0.59, 0.29, 0.0}
\definecolor{beaublue}{rgb}{0.74, 0.83, 0.9}
\definecolor{orange}{rgb}{1.0, 0.5, 0.0}
\definecolor{darkslategray}{rgb}{0.18, 0.31, 0.31}
\definecolor{deepblue}{rgb}{0,0,0.5}
\definecolor{deepred}{rgb}{0.6,0,0}
\definecolor{deepgreen}{rgb}{0,0.5,0}
\lstdefinestyle{myMatlabstyle}{
	language=Matlab,
	backgroundcolor=\color{white},
	commentstyle=\color{codegreen},
	keywordstyle=\color{blue},
	%identifierstyle=\color{brown},
	numberstyle=\tiny\color{codegray},
	stringstyle=\color{orange},
	basicstyle=\footnotesize,
	breakatwhitespace=false,
	breaklines=true,
	captionpos=b,
	keepspaces=true,
	numbers=left,
	numbersep=5pt,
	showspaces=false,
	showstringspaces=false,
	showtabs=false,
	tabsize=2
}
\lstdefinestyle{myPythonstyle}{
	language=Python,
	basicstyle=\ttfamily\small,
	keywordstyle=\color{blue},
	backgroundcolor=\color{white},
	commentstyle=\color{green},
	stringstyle=\color{red},
	showstringspaces=false,
	%identifierstyle=\color{brown},
	breaklines=true,
}
\lstset{language=Matlab,frame=single}
\lstset{language=Python,frame=single}
\usepackage{amsmath}
\usepackage{epsfig}         % to insert PostScript figures
       % to insert PostScript figures

\begin{document}


%          Definitions of commonly used symbols.



%          The title and header.

\noindent
{\scriptsize ME 422, Fall 2018} \hfill

\begin{center}
\large
Assignment 3.
\normalsize

Jithin D. George
\end{center}

\noindent
Due Oct 24
\vspace{.3in}

%           The questions!



\noindent


\begin{enumerate}
\item



\begin{enumerate}
\item
In this example, the temperature is a constant, say $T_0$.

Thus, the temperature at any time could be described relative to it.
\[T(t) = T^* T_0\]
Thus,
\[T^* = \frac{T(t)}{T_0}\]

\item

\[\beta = \frac{1}{k_B T}\]
Now, $k_B T_0$ is a constant. So, we define
\[\beta^* = k_B T_0 \beta = \frac{1}{T^*}\]

\item 
\[\Delta E = \beta \epsilon =\frac{1}{k_B T} \epsilon \]
\[\Delta E^*  = \frac{k_B T_0}{\epsilon} \frac{1}{k_B T} \epsilon \]



\end{enumerate}


\item

\begin{enumerate}
\item
For the 5 temperatures (1,5,10,15,20), the ratio is plotted below.

\item 
As the temperature increases, the probability of switching goes to 1. This is because the energy $\beta \epsilon$ goes to zero with increasing temperature. Thus, the random variable in the Metropolis algorithm will always be less than $e^0$ and switching would happen nearly always.

\end{enumerate}

\item 

\begin{enumerate}
\item 
\begin{enumerate}
\item
magnetization measures the number of positive spins minus the number of negative spins.

\item 
Average magnetization measures the average magnetization over nsteps.

\end{enumerate}
\item
 The minimum prossible energy is  $-4  \epsilon$.

 \item

 \begin{tikzpicture}




 \draw[->] (3.5,-0.7)-- (3.5,0) ;
 \draw[->](4,-0.7) -- (4,0) ;
 \node at (5,-0.7) { and };
 \draw[->](6,-1) -- (6,-1.7) ;
 \draw[->] (6.5,-1) --(6.5,-1.7)  ;


 \draw[->] (4,-1.7)--(4,-1)  ;
  \draw[->](3.5,-1.7) -- (3.5,-1) ;

	\draw[->](6.5,0) -- (6.5,-0.7) ;
  \draw[->](6,0)-- (6,-0.7)  ;
 \end{tikzpicture}

\end{enumerate}

\item
Magnetization is lost on increasing temperature.
Iron is magnetic at room temperature. So,
\[\frac{\epsilon}{k_B T_r} >>0\]
So, we would expect 
\[\frac{\epsilon}{k_B T_r} > 1\]
\[\alpha k_B T_r >   k_B T_r\]
So, 
\[\alpha  >   1\]

Let's see if it's true. Iron loses magnetization at 1024 K.
So,
\[\frac{\epsilon}{k_B 1024} \approx 0\]
\[\frac{\epsilon}{k_B 1024} < 1\]
\[\epsilon  < 1.4 *10^{-20}J\]
We would guess $\epsilon$ around $1.4 *10^{-20}$ which is still bigger than $k_B T_r$

So, $\alpha$ should be greater than 1.
\end{enumerate}
\end{document}
